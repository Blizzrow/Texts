\section{Anhang}

Code für PICKIT: 

\begin{lstlisting}{language = [Sharp]C}
using System;
using System.IO;
using System.Text;
using System.Threading;
using System.Threading.Tasks;
using PICkitS;

namespace Bachelor
{
    class Program
    {
        static double Calc_Temp(byte TEMP_HIGH, byte TEMP_LOW)
        {
            int New_High = Convert.ToInt32(TEMP_HIGH);
            int New_Low = Convert.ToInt32(TEMP_LOW);
            int New_High = Convert.ToInt16(TEMP_HIGH);
            int New_Low = Convert.ToInt16(TEMP_LOW);
            New_High = New_High << 8;
            double Temp = New_High + New_Low;
            Temp = ((Temp / 1024 * 5) - 0.4) / 0.0195;
            return Temp;

        }

        static double Calc_I_Out(byte I_Out_High, byte I_Out_Low)
        {
            int New_I_Out_High = Convert.ToInt32(I_Out_High);
            int New_I_Out_Low = Convert.ToInt32(I_Out_Low);
            int New_I_Out_High = Convert.ToInt16(I_Out_High);
            int New_I_Out_Low = Convert.ToInt16(I_Out_Low);

            New_I_Out_High = New_I_Out_High << 8;
            double Amps = New_I_Out_High + New_I_Out_Low;
            Amps = Amps / 1024 * 5 / 0.3;
            return Amps;


        }

        static double Calc_V_Out(byte V_Out_High, byte V_Out_Low)
        {
            int New_V_Out_High = Convert.ToInt32(V_Out_High);
            int New_V_Out_Low = Convert.ToInt32(V_Out_Low);
            int New_V_Out_High = Convert.ToInt16(V_Out_High);
            int New_V_Out_Low = Convert.ToInt16(V_Out_Low);

            New_V_Out_High = New_V_Out_High << 8;
            double Volts = New_V_Out_High + New_V_Out_Low;
            Volts = Volts / 1024 * 5 * 6;
            return Volts;
        }


        static double Calc_V_In(byte V_In_High, byte V_In_Low)
        {
            int New_V_In_High = Convert.ToInt32(V_In_High);
            int New_V_In_Low = Convert.ToInt32(V_In_Low);
            int New_V_In_High = Convert.ToInt16(V_In_High);
            int New_V_In_Low = Convert.ToInt16(V_In_Low);

            New_V_In_High = New_V_In_High << 8;
            double Volts = New_V_In_High + New_V_In_Low;
            Volts = Volts / 1024 * 5 * 13;
            return Volts;
        }




        static void Main(string[] args)
        {
            /*Slave Adress of the Device */
            byte DEV_ADDR_WRITE = 0x10;


            /*Number of Bytes that should be read*/
            byte NUM_BYTES = 0x08;

            /*Registers that I want to read*/
            byte V_IN_ADDR = 0x20;
            byte I_OUT_ADDR = 0x22;
            byte V_OUT_ADDR = 0x24;
            byte TEMP_ADDR = 0x26;

            /*Array where the data gets saved*/
            byte[] Data_Array = new byte[NUM_BYTES];
            Data_Array[0] = 0x20;

            string Script_View = "";

            /*Set Clock speed to 100kHz */
            int Set_Clock = 100;
            double Voltage = 3.3;

            /*Timeout after 2 seconds*/
            int Set_TimeOut = 10000;

            string now = DateTime.Now.ToString("h:mm:ss tt");
            string path = @"C:\Users\lamer\Downloads\Test.txt";
            /*Set Clock speed to 100kHz  and voltage to 3.3 V*/
            int Set_Clock = 400;
            double Voltage = 3.3;

            /*Timeout after 10 seconds*/
            int Set_TimeOut = 100;
       
            string now = DateTime.Now.ToString("h:mm:ss tt");
            string path = @"C:\Users\lamer\Downloads\Board_2_Test_3_Parallel.txt";

            byte High = 0x01;
            byte Low = 0x54;
            double x = Calc_Temp(High, Low);
            Console.WriteLine(x);
            int new_high = Convert.ToInt32(High) << 8;
            int net = new_high + Convert.ToInt32(Low);
            Console.WriteLine(new_high);
            Console.WriteLine(Low);
            Console.WriteLine(net);

            /*Initialisiert den PICKIT, falls mehrere Analyzer verwendet werden, 
            dann die Funktion  Initialize_PICkitSerial mit dem USB Index verwenden 
            Hier ist der USB-Index == 0*/
            if (Device.Initialize_PICkitSerial(0))
            {
                Console.WriteLine("Device Initialization successful");
            }
            else
            {
                Console.WriteLine("Device Initialization failed \n");
                return;
            }
            /*PicKit für I2C Master konfigurieren */
            if (I2CM.Configure_PICkitSerial_For_I2CMaster())
            {
                Console.WriteLine("I2C Master Initialization successful");
            }
            else
            {
                Console.WriteLine("I2C Master Initialization failed \n");
                return;
            }

            if (I2CM.Set_I2C_Bit_Rate(Set_Clock))
            {
                Console.WriteLine("Clock Set successfully to: " + Convert.ToString(Set_Clock));
            }
            else
            {
                Console.WriteLine("Setting Clock failed \n");
                return;
            }

            I2CM.Set_Read_Wait_Time(Set_TimeOut);

            Console.WriteLine("Timeout set to: " + Convert.ToString(Set_TimeOut) + " milliseconds\n");

            //Netzteil hat interne Pull-Ups, daher keine PicKit pullups anmachen        
            if (I2CM.Set_Pullup_State(false))
            {
                Console.WriteLine("Pullups Enabled");
            }
            else
            {
                Console.WriteLine("Pullups could not be enabled");
            }
          
            while (1)
            {
                I2CM.Read(DEV_ADDR_WRITE, V_IN_ADDR, NUM_BYTES, ref Data_Array, ref Script_View);
                double V_IN = Calc_V_In(Data_Array[0], Data_Array[1]);
                double I_OUT = Calc_I_Out(Data_Array[2], Data_Array[3]);
                double V_OUT = Calc_V_Out(Data_Array[4], Data_Array[5]);
                double Temp = Calc_Temp(Data_Array[6], Data_Array[7]);

                now = DateTime.Now.ToString("h:mm:ss");

                string Temp1 = Temp.ToString("0.00");
                string V_OUT1 = V_OUT.ToString("0.00");
                string V_IN1 = V_IN.ToString("0.00");
                string I_OUT1 = I_OUT.ToString("0.00");

                using (StreamWriter w = File.AppendText(path))
                {
                    w.WriteLine(now + " " + Temp1 + " " + V_OUT1 + " " + V_IN1 + " " + I_OUT1);
                }






            }
        }
    }

\end{lstlisting}

\newpage

1. Variante der Datenabfrage: \\

\begin{lstlisting}[language = Python]
def rec_data(device_adress, data_to_send, dcount):
  temp = bus.read_i2c_block_data(device_adress, TEMP_ADDR,2)
  i_out = bus.read_i2c_block_data(device_adress, I_OUT_ADDR,2)
  v_out = bus.read_i2c_block_data(device_adress, V_OUT_ADDR,2)
  v_in = bus.read_i2c_block_data(device_adress, V_IN_ADDR,2)

  temp_high = temp[0]
  temp_low= temp[1]

  i_high = i_out[0]
  i_low = i_out[1]

  v_out_high = v_out[0]
  v_out_low = v_out[1]

  v_in_high = v_in[0]
  v_in_low = v_in[1]

\end{lstlisting}


2. Variante der Datenabfrage: \\
\begin{lstlisting}[language = Python]
from smbus2 import SMBus
import numpy as np
import time
import matplotlib.pyplot as plt
from soundplayer import SoundPlayer
from datetime import datetime
import sys

DEV_ADDR1 = 0x08
V_IN_ADDR = 0x20
I_OUT_ADDR = 0x22
V_OUT_ADDR = 0x24
TEMP_ADDR = 0x26

V_IN_HIST = []
I_OUT_HIST = []
V_OUT_HIST = []
TEMP_HIST = []
POW_HIST = []
data_to_send = np.empty((50))
dcount = 0


def calc_temp(val_high, val_low):
    val = (val_high << 8) + val_low
    Temp = ((val /1024 * 5) -0.4) / 0.0195
    return Temp

def calc_i_out(val_high, val_low):
    val = (val_high << 8) + val_low
    Amps = (val /1024 * 5) / 0.3
    return Amps

def calc_v_out(val_high, val_low):
    val = (val_high << 8) + val_low
    Volt = (val /1024 * 5) * 6
    return Volt


def calc_v_in(val_high, val_low):
    val = (val_high << 8) + val_low
    Volt = (val /1024 * 5) * 13
    return Volt

def rec_data(device_adress, data_to_send, dcount):

  Data = bus.read_i2c_block_data(device_adress, V_IN_ADDR,8)

  temp_high = Data[6]
  temp_low= Data[7]

  i_high = Data[2]
  i_low = Data[3]

  v_out_high = Data[4]
  v_out_low = Data[5]

  v_in_high = Data[0]
  v_in_low = Data[1]


  new_row = []
  current_time = datetime.now().strftime("%H:%M:%S.%f")
  new_row.append(current_time)
  TEMP = calc_temp(temp_high, temp_low)

  new_row.append(TEMP)

  I_OUT = calc_i_out(i_high, i_low)
  new_row.append(I_OUT)

  V_OUT = calc_v_out(v_out_high, v_out_low)
  new_row.append(V_OUT)

  V_IN = calc_v_in(v_in_high, v_in_low)
  new_row.append(V_IN)

  new_row.append(V_OUT*I_OUT)
 
  return new_row

bus = SMBus(1)
while(1):
  try:
     Data1 = rec_data(DEV_ADDR1)
     with open("Test.txt", 'a') as f:
            f.write(str(Data1) + "\n")
   
  except KeyboardInterrupt:
        bus.close()
        sys.exit()
plt.show()

\end{lstlisting}

\newpage
Code für die Datenverarbeitung  \\

\begin{lstlisting}[language = Python]
import numpy as np
import math
import matplotlib.pyplot as plt
import mpld3
import scipy.signal as scys
import enum
from datetime import datetime
import os

def get_measure_time(t1, t2, sample_cnt):
    fmt = '%H:%M:%S:%f'
    t1 = t1#[:-4]
    t2 = t2#[:-4]
    tstamp1 = datetime.strptime(t1,fmt)
    tstamp2 = datetime.strptime(t2,fmt)

    if tstamp1 > tstamp2:
        td = tstamp1 - tstamp2
    else:
        td = tstamp2 - tstamp1
    td_secs = int(round(td.total_seconds()))
    print(td_secs)
    time = []
    print(sample_cnt/td_secs)
    for i in range(0, sample_cnt):
        x = i * td_secs / sample_cnt
        time.append(x)

    with open("Metadata.txt" ,"a") as f:
            f.write(str(sample_cnt/td_secs) + "\n")
    return time

#Enumeration, hat Atm keine Funktionalität
class labels(enum.Enum):
        Temperatur = 1
        Ausgangsspannung = 2
        Eingangsspannung = 3
        Ausgangsstrom = 4
        Leistung = 5
verz = [" in °Celsius", " in Volt", " in Volt", " in Ampere", " in Watt"]

#Metadaten für alle Parameter (Spannung,Strom,Temp) berechnen und in einem 2D Array speichern
def calc_meta(data):
        daten = np.empty((4, 5))
        for i in range(0,4):
            Mittel = np.mean(data[i])
            StandAbw = np.std(data[i])
            len_data = len(data[:][0])
            minimum = np.min(data[i])
            maximum = np.max(data[i])
            daten[i][0] = Mittel
            daten[i][1] = StandAbw
            daten[i][2] = len_data
            daten[i][3] = minimum
            daten[i][4] = maximum
            print("Mittelwert: ", Mittel,  "Standardabweichung: ", StandAbw,
               "Länge: " , len_data,  "Minimum: ", minimum,  "Maximum: ", maximum)
        return daten

#Metadaten z. B. nur für die Temperatur von einem Board berechnen
def calc_meta_single(data):
    Mittel = np.mean(data)
    StandAbw = np.std(data)
    len_data = len(data)
    minimum = np.min(data)
    maximum = np.max(data)
    return Mittel, StandAbw, len_data, minimum, maximum

#Entpacken der Daten aus der Textfile und aufteilen in Zeitverlauf und Daten
def unr_data(path):
        count_samples = 0
        count_errors = 0
        data = np.genfromtxt(path, dtype ='str')
        Daten = np.empty((5, len(data)))

       # with open("Metadata.txt" ,"a") as f:
         #   f.write(path+" ")
        Time = data[:,0]

        #Kleiner Loop um Fehler beim Format zu beseitigen
        #z. B. 08:14:54.56666 -> 08:14:54:56666
        #der Punkt ist ein Fehler der mir beim Daten generieren passiert ist
        for idx,element in enumerate(Time):
            element = np.char.replace(element, '.', ':')
            Time[idx] = element
        Time = get_measure_time(str(Time[0]), str(Time[len(data) -1]), len(data))

        Temp = data[:,1]
        Temp = Temp.astype(np.float32)

        V_OUT = data[:,2]
        V_OUT = V_OUT.astype(np.float32)

        V_IN = data[:,3]
        V_IN = V_IN.astype(np.float32)

        I_OUT = data[:,4]
        I_OUT = I_OUT.astype(np.float32)

        POW = V_OUT * I_OUT

        count_samples += len(data)

        Daten[0] = Temp
        Daten[1] = V_OUT
        Daten[2] = V_IN
        Daten[3] = I_OUT
        Daten[4] = POW

        for i in range(0, 4):
            for j in range(0, len(data)):
                if(Daten[i][j] < 0 or Daten[i][j] > 100):
                    count_errors+=1
                   # Daten[i][j] = Daten[i][j-1]
        print(count_errors)

        return Daten, Time

#Funktion zum Filtern von hohen Frequenzen
def apply_filter(data):
        n = 15
        b = [1.0 / n] * n
        a = 1
        filtered = scys.lfilter(b,a, data)
        return filtered

#Berechnen der Leistung aus Daten
def Calc_Power(Voltage, Current):
        return Voltage * Current


#War eine spielerei, hat at the moment keine Verwendung
def Conf_Plot(Data = None, Time = None, x = None, y = None, Filter = None, Title = None):
        if(Filter == True):
            Data = apply_filter(Data)

        y = labels[y].value
        fig = plt.figure()
        ax1 = fig.add_subplot(211)
        if(x == None):
            ax1.set_ylabel(labels(y).name + verz[y-1])
            ax1.set_xlabel("Zeit in Sekunden")
            ax1.set_title(Title)
            y_mean = [np.mean(Data[y-1])] * len(Data[y-1])
            ax1.plot(Time[:], Data[y-1])#, label= labels(y).name + verz[y] +  vs  + Zeit in Sekunden)
        else:
            x = labels[x].value
            ax1.set_ylabel(labels(y).name + verz[y-1])
            ax1.set_xlabel(labels(x).name + verz[x-1])
            ax1.set_Title(Title)
            ax1.plot(Data[x-1], Data[y-1])#, label= labels(y).name + verz[y-1] +  vs  + labels(x).name + verz[x-1].name)
        ax1.legend(loc ='best')
        plt.savefig(labels(y).name)
        plt.plot()
        plt.show()
        
directory = 'C:\\Users\\lamer\\Downloads\\Bachelor\\Data\\CSV\\'
for filename in os.listdir(directory):

    Data, Time = unr_data(directory+filename)
    x = calc_meta_single(Data[4])
    print(filename,": ", x[0], x[1])

    plt.plot(Time,Data[0], label = filename)
    plt.xlabel('Zeit in Sekunden')
    plt.ylabel('Temperatur in C')
    plt.legend(loc = 2)
    plt.show()

\end{lstlisting}




\newpage

Code für das Neuronale Netz. 
\begin{lstlisting}[language = Python]
import tensorflow as tf
from tensorflow import keras
from tensorflow.keras import layers
import pydot 
import graphviz
import numpy as np
import math
import matplotlib.pyplot as plt
import mpld3
import scipy.signal as scys
import enum
from datetime import datetime

config = tf.compat.v1.ConfigProto()
config.gpu_options.allow_growth = True
config.log_device_placement = True
sess= tf.compat.v1.Session(config=config)

Training_Data = np.concatenate((_5Hz, _20Hz, _30Hz,  __5Hz, __20Hz, __30Hz,___5Hz, ___20Hz, ___30Hz), axis =0)
np.random.shuffle(Training_Data)
Solution = Training_Data[:, Training_Data.shape[1]-1]
Training_Data = Training_Data[:,:-1]
Input_Parameter = Training_Data.shape[1]
Output_Layer_Size = 3
Hidden_Layer_Size = int(Input_Parameter*(2/3)) + Output_Layer_Size
print(Hidden_Layer_Size)



for i in range(0, Training_Data.shape[1]-2):
    max1 = max(Training_Data[:,i])
    min1 = min(Training_Data[:,i])
    Training_Data[:,i] = ( (Training_Data[:,i] - min1) / (max1-min1))
    

with sess:
    model = tf.keras.Sequential(
        [
            layers.Dense(80, input_dim = Input_Parameter,  activation ="relu", name = "Start"),
            layers.Dense(40, activation = "relu", name = "hidden1"),
            layers.Dense(3, activation ="softmax", name = "Output"),
        ]
     )
    model.compile(optimizer = "adam", loss=tf.keras.losses.SparseCategoricalCrossentropy(from_logits=True),
                      metrics=['accuracy'])
    history = model.fit(Training_Data, Solution, epochs = 300)
   
\end{lstlisting}
