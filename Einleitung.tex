\section{Einleitung}
\begin{flushleft}
Buck Wandler sind heutzutage in aller Munde, sie stellen beispielsweise Spannung in Notebook Prozessoren und Ladegeräten zu verfügung, Regeln den Strom an Stepper-Motoren. Durch die weitläufigen Anwendungsfälle, ist es besonders in der heutigen Zeit, in der Künstliche Intelligenz und Machine Learning Algorithmen immer wichtiger werden, zu evaluieren, inwieweit sich solche Wandler für Machine Learning Anwendungen eignen, wenn Daten über Strom, Spannung, Leistung und Temperatur bekannt sind. Deshalb ist das Thema dieser Arbeit die Analyse eines Hybrid-Buck-Wandlers. Im Folgenden wird solch ein Wandler hinsichtlich seiner Anwendbarkeit für Themen im Bereich Maschine Learning und Zustandsüberwachung analysiert werden. Die Analyse erfolgt dabei in mehreren Schritten. Zuerst werden Daten erhoben, während der Wandler an unterschiedliche statische Lasten angeschlossen ist, um einen Überblick über das Verhalten des Wandlers zu gewinnen. Anschließend wird der Wandler an eine dynamische Last angeschlossen und es werden periodische Signale gemessen. Die gemessenen Daten werden dann in Echtzeit mit bereits bekannten Daten verglichen und ebenfalls gegen andere, parallel laufende Wandler, abgestimmt. Das Letztendliche Ziel dieser Arbeit ist es, zu evaluieren, wie präzise und zuverlässig die Daten sind und inwieweit diese sich für Anwendungen in den am Anfang genannten Bereichen eignen.\\

Da der vorliegende Wandler über einen Mikrocontroller mit I2C Schnittstelle verfügt, aus welchem Strom, Spannungs und Temperaturdaten erhoben werden können, wird das Erfassen der Daten durch das vorhandene Protokoll stark vereinfacht und sehr portierbar gemacht, was es ermöglicht die Daten aus beliebigen Platformen (Win 10, Linux, etc.) oder einfach durch andere Mikrocontroller (z.B. Raspberry Pi) auszulesen.  



\end{flushleft}