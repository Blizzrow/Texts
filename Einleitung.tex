\section{Einleitung}

DC-DC-Buck-Wandler, oder auch Tiefsetzsteller, sind heutzutage Bestandteil verschiedenster Applikationen. So stellen diese beispielsweise Spannung in Notebook-Prozessoren und Ladegeräten zu Verfügung oder regeln den Strom an Stepper-Motoren. Durch die große Anzahl an Anwendungsfällen, ist es besonders in der heutigen Zeit, in der Künstliche Intelligenz und die damit einhergehenden Machine Learning Algorithmen immer wichtiger und präsenter werden, zu evaluieren, inwieweit sich solche Wandler für Machine Learning Anwendungen eignen, wenn Daten über Strom, Spannung, Leistung und Temperatur des Wandlers über eine Digitale Schnittstelle ausgelesen werden können. Deshalb beschäftigt sich diesee Arbeit die Analyse eines Hybrid-Buck-Wandlers für Anwendungen im Bereich des Maschine Learning bzw. der künstlichen Intelligenz.
\\
\\Dadurch bedingt, dass es sich hierbei um einen neu entwickelten Wandler handelt, welcher noch nicht auf dem Markt ist und sich beim Schreiben dieser Arbeit noch in der Testphase befindet, wird die Analyse in mehreren Schritten durchgeführt werden. Zuerst werden in dieser Arbeit allgemeine Testfälle durchgeführt, um die Qualität der Daten sicherzustellen und um einen Einblick in das Verhalten des Wandlers zu gewinnen. Neben der Qualität der Daten muss wird in dieser Arbeit ebenfalls die Quantität der Daten erfasst, dies bedeutet im speziellen, wie viele Daten mit diesem Wandler in einer Zeiteinheit generiert werden können. Des Weiteren wird im Zusammenhang mit dieser Arbeit Software generiert, mit dem die vorhandenen Daten ausgelesen, manipuliert und visualisiert werden können. Schlussendlich ist das Ziel dieser Arbeit, einen ersten Test bezüglich der Anwendbarkeit und Zuverlässigkeit der vom Buck-Wandler generierten Daten für Maschine Learning Applikationen, im speziellen dem der Neuronalen Netze, durchzuführen und zu bewerten.


