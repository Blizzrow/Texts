\section*{Abstract}

Die Anwendung von Maschine Learning Algorithmen auf Daten, welche von Hybrid-Buck-Wandler stammen ist momentan nicht weit in der Industrie ausgeprägt. Aus diesem Grund beschäftigt sich diese Arbeit mit der Analyse des o. g. Wandler-Typen hinsichtlich seiner Anwendbarkeit für Maschine Learning Applikationen. Die Analyse beinhaltet die Generierung von Daten, eine Analyse bzgl. der Qualität der Daten sowie der maximalen Abtastrate. Abgeschlossen wird die Analyse mit einem Testfall in Form eines Neuronalen Netzes. Die Ergebnisse der Analyse zeigten, dass die Daten nur geringe Abweichungen von ihrem Mittelwert haben, sowie dass die maximale Abtastrate etwa \hl{Hz} beträgt. Die Anwendung von Neuronalen Netzen anhand von verschiedene periodischen Signalen zeigte eine Genauigkeit von \hl{x}. Zusammenfassend besteht für die in dieser Arbeit behandelten Hybrid-Buck-Wandler Potential für ML Applikationen verwendet zu werden, insofern es sich bei dem Leistungssignal um ein niederfrequentes Signal handelt, da die Daten sonst durch den Alias Effekt unbrauchbar gemacht werden. 